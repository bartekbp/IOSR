\documentclass[a4paper,11pt]{article}
\usepackage[english,polish]{babel}
\usepackage{polski}
\usepackage[utf8]{inputenc}
\usepackage{graphicx}
\usepackage{tikz-uml}
\usepackage{tikz} 

\usepackage{pgf}
\usepackage{listings}
\usepackage{color}
\usepackage{tabularx}
\usepackage{amsmath}
\newcommand{\HRule}{\rule{\linewidth}{0.5mm}}
\begin{document}
\begin{titlepage}
    \begin{center}
        \includegraphics[width=0.4\textwidth]{images/logo.jpg} \\[1cm]
        \textsc{\LARGE Inżynieria Oprogramowania Systemów Rozproszonych} \\[0.8cm]
        \textsc{\LARGE Dokumentacja} \\[0.5cm]
        \HRule \\[0.4cm]
        { \huge \bfseries Kaflog} \\[0.4cm]
        \HRule \\[1.5cm]
    
    \begin{minipage}{0.4\textwidth}
        \begin{flushleft} \large
        \emph{Autorzy:} \\
        Łukasz \textsc{Opioła} \\
        Bartosz \textsc{Polnik} \\
        Michał \textsc{Janiec}
        \end{flushleft}
    \end{minipage}
    \begin{minipage}{0.4\textwidth}
        \begin{flushright} \large
            \emph{Prowadzący:} \\
            Dr. Inż. Marcin \textsc{Jarząb}
        \end{flushright}
    \end{minipage}

    \vfill

    {\large \today}

    \end{center}
\end{titlepage}

\section{Opis tematu}
    Celem projektu jest utworzenie systemu agregacji logów z rozproszonych węzłów obliczeniowych wspierających masywną skalowalność w oparciu o bibliotekę Kafka. Na węzłach uruchomieni są agenci (producenci) śledzący zmiany w plikach zdarzeń i wysyłający informacje do brokera Kafka, który następnie wpisuje dane do Hadoopa. Celem jest dostarczenie określonych raportów odnośnie kategorii zdarzeń (INFO, ERROR, DEBUG), częstotliwości ich występowania etc. Utworzone rozwiązanie będzie realizować architekturę lambda.

\section{Technologie} 
    Projekt zostanie zrealizowany z wykorzystaniem następujących technologii:
    \begin{itemize}
        \item Cloudera - do konfiguracji i zarządzania clustrem
        \item Hadoop Yarn (MR2) - do importu danych z brokera Kafka
        \item Hive - do przechowywania danych umożliwiających generowanie widokow
        \item HBase - do przechowywania widoków
        \item Kafka - do zbierania logów od agentów
        \item Impala - do łączenia widoków warstwy usługowej
        \item JMX - do monitorowania agentów
    \end{itemize}

\section{Przypadki użycia}

\begin{tikzpicture} 
    \begin{umlsystem}[x=0, y=0, fill=red!10]{KafLog} 
        \umlactor[x=-7.5]{Użytkownik}

        \foreach \x/\a in {2/Zarejestruj agenta, 1/Pokaż stan systemu, 0/Pokaż statystyki agenta, -1/Pokaż statystyki zbiorcze, -2/Pokaż historię logów} {
            \umlusecase[name=\x,x=0, y=\x]{\a}; 
            \umlassoc{Użytkownik}{\x};
        }
    \end{umlsystem} 
\end{tikzpicture}

\section{Podział zadań}

\begin{tabularx}{\linewidth}{ X | X | X}
  \textsc{Łukasz Opioła} & \textsc{Bartosz Polnik} & \textsc{Michał Janiec} \\ \hline
  utworzenie szablonu środowiska & zarządzanie zespołem & zbieranie logów w celu przetwarzania real-time \\ \hline
  tworzenie use-case & utworzenie szablonu projektu i podział na moduły & przetwarzanie logów w czasie rzeczywistym \\ \hline
  utworzenie producenta & zapisywanie logów w celu generacji widoków w hdfsie & utworzenie inkrementacyjnych widoków na podstawie przetwarzanych danych \\ \hline
  generowanie statystyk przez agentów & utworzenie widoków warstwy batch & utworzenie widoku statystyk w gui \\ \hline
  widok w gui do monitorowania agentów & łączenie widoków & dodanie komentarzy do kodu \\ \hline
  widok w gui aktualnie generowanych logów & testowanie platformy & refactoring kodu \\ \hline
  testy wydajnościowe & dopasowanie generacji widoków real-time i batch celem integracji z Apache Impala & optymalizacja działanie aplikacji do monitorowania agentów \\ \hline
  zapewnienie bezpieczeństwa platformie & dodanie bezpieczeństwa do aplikacji monitorującej agentów & dodanie informacji o przedziale czasowym statystyk w gui\\ \hline
  opis poszczególnych modułów & utworzenie przykładów dla gui & umożliwienie uruchamiania storma lokalnie \\ \hline
  utworzenie diagramu architektury & uruchamianie cykliczne generacji widoków warstwy batch & poprawia wizualna widoku statystyk\\ \hline
\end{tabularx}




\end{document}


%% TODO HANDLE: kreślenie tematu projektu wraz z opisem proponowanego problemu,
