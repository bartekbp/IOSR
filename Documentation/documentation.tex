\documentclass[a4paper,11pt]{article}
\usepackage[english,polish]{babel}
\usepackage{polski}
\usepackage[utf8]{inputenc}
\usepackage{graphicx}
\usepackage{listings}
\usepackage{color}
\usepackage{amsmath}
\newcommand{\HRule}{\rule{\linewidth}{0.5mm}}
\begin{document}
\begin{titlepage}
    \begin{center}
        \includegraphics[width=0.4\textwidth]{images/logo.jpg} \\[1cm]
        \textsc{\LARGE Inżynieria Oprogramowania Systemów Rozproszonych} \\[0.8cm]
        \textsc{\LARGE Dokumentacja} \\[0.5cm]
        \HRule \\[0.4cm]
        { \huge \bfseries System agregacji i przetwarzania zdarzeń systemowych} \\[0.4cm]
        \HRule \\[1.5cm]
    
    \begin{minipage}{0.4\textwidth}
        \begin{flushleft} \large
        \emph{Autorzy:} \\
        Łukasz \textsc{Opioła} \\
        Bartosz \textsc{Polnik}
        \end{flushleft}
    \end{minipage}
    \begin{minipage}{0.4\textwidth}
        \begin{flushright} \large
            \emph{Prowadzący:} \\
            Dr. Inż. Marcin \textsc{Jarząb}
        \end{flushright}
    \end{minipage}

    \vfill

    {\large \today}

    \end{center}
\end{titlepage}

\section{Opis tematu}
    System agregacji logów z rozproszonych węzłów obliczeniowych wspierających masywną skalowalność w oparciu o bibliotekę Kafka. Na węzłach uruchomieni są agenci (producenci) śledzący zmiany w plikach zdarzeń i wysyłający informacje do brokera Kafka, który następnie wpisuje dane do Hadoopa. Celem jest dostarczenie określonych raportów odnośnie kategorii zdarzeń (INFO, ERROR, DEBUG), częstotliwości ich występowania etc.

\section{Technologie} 
    Projekt zostanie zrealizowany z wykorzystaniem następujących technologii:
    \begin{itemize}
        \item Cloudera
        \item Hadoop Yarn (MR2)
        \item Hive
        \item HBase
        \item Oozie
        \item Kafka
    \end{itemize}

\section{Podział zadań}

\begin{tabular}{ c | c }
  Łukasz Opioła & Bartosz Polnik \\ \hline
  4 & 5 \\
  7 & 8 \\
\end{tabular}


\end{document}
